\documentclass[12pt]{article}

% Packages for formatting and math
\usepackage{amsmath, amsthm, amssymb}
\usepackage{cite}
\usepackage{geometry}
\geometry{a4paper, margin=2.5cm}


% Theorem and Proof setup
\newtheorem{theorem}{Theorem}
\newtheorem{lemma}{Lemma}

% Title Section
\title{MATD02 - Exploration of Finite Geometry }
\author{Your Name}
\date{}

\begin{document}

    \maketitle

    \begin{abstract}
        Lorem ipsum dolor sit amet, consectetur adipiscing elit.
        Pellentesque pretium ut neque ac lacinia.
        Aenean ultricies dapibus ultricies.
        Nam tristique posuere quam, eget varius lorem posuere a.
        Nam hendrerit magna ex, vehicula tempor quam maximus at.
        Lorem ipsum dolor sit amet, consectetur adipiscing elit.
        Nullam vitae ultrices nulla, at dapibus lectus.
        Nullam ultrices eros eu lorem laoreet elementum.
        Nullam ullamcorper in lectus vel varius.
        Praesent lobortis sem vitae elit posuere, in consectetur magna tempor.
        Vivamus nunc ligula, ullamcorper ac sapien vel, dignissim varius leo.
        Mauris vel suscipit metus, id ornare odio.
        Vivamus ultricies vehicula mi ut bibendum.
        Nullam sollicitudin elit sit amet velit lacinia, eget pretium nisl viverra.
        Nulla laoreet nisi non velit posuere, a vehicula diam vehicula.
        Donec fermentum commodo tortor, ac feugiat orci iaculis sit amet.
        Suspendisse luctus, velit eu placerat dapibus, justo felis mattis tortor, vitae blandit leo justo vitae nibh.
    \end{abstract}


    \section{Introduction}

    Introduce four/five point geometries as a way to start thinking about geometries more abstractly, use notes from\cite{sinn_course_2021}

    \section{The Fano plane}

    \begin{theorem}
        Ex. Each point is on exactly three lines
    \end{theorem}

    \begin{proof}

    \end{proof}

    \section{Connections to finite groups}
    Explore ideas in from the following sources\cite{carmichael_finite_1930,tsuzuku_finite_1982},
    I'd really like to look at broader connections than symmetry groups.

    \begin{theorem}
    \end{theorem}

    \begin{proof}
    \end{proof}

    \section{Future Developments}

    Take a look at notes from the group theory textbooks.

    % References
    \bibliography{references}
    \bibliographystyle{IEEEtran}
\end{document}
