\documentclass[12pt]{article}

% Packages for formatting and math
\usepackage{amsmath, amsthm, amssymb}
\usepackage{cite}
\usepackage{geometry}
\geometry{a4paper, margin=2.5cm}


% Theorem and Proof setup
\newtheorem{theorem}{Theorem}
\newtheorem{lemma}{Lemma}

% Title Section
\title{MATD02 - Exploration of Finite Geometry }
\author{Your Name}
\date{}

\begin{document}

    \maketitle

    \begin{abstract}
        This is where you write a brief paragraph summarizing the main idea of your topic and your approach. Keep it concise and clear, offering a high-level view of your paper.
    \end{abstract}


    \section{Introduction}
    Begin your report by introducing the topic. Clearly define the key concepts, notations, and theorems that will be discussed. Keep the language accessible to the entire MATD02 class.

    \section{Theorems and Proofs}

    \begin{theorem}
        State the first theorem here.
    \end{theorem}

    \begin{proof}
        Provide the proof for Theorem 1 here, step by step.
    \end{proof}

    \begin{theorem}
        State the second theorem here.
    \end{theorem}

    \begin{proof}
        Provide the proof for Theorem 2 here, ensuring clarity and logical flow.
    \end{proof}

    \section{Future Developments}
    In this section, discuss any potential future developments related to your topic. Mention related problems, extensions, or more advanced ideas, even if their proofs are not currently clear.

    \section{Conclusion}
    Summarize the key points covered in the report, including the theorems and their significance. Restate the main takeaways in a clear, concise manner.

    % References
    \bibliography{references}
    \bibliographystyle{IEEEtran}
\end{document}


%\begin{document}
%    When you are not looking at it, this sentences stops citing~\cite{knuth1990}.
%
%    \bibliography{references}
%    \bibliographystyle{plain}
%\end{document}
