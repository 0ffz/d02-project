\documentclass[12pt]{article}

% Packages for formatting and math
\usepackage{amsmath, amsthm, amssymb}
\usepackage{cite}
\usepackage{geometry}
\usepackage{csquotes}
\geometry{a4paper, margin=2.5cm}

%\setlength{\parskip}{\baselineskip}%
%\setlength{\parindent}{0pt}
\usepackage{parskip}
\setlength{\parindent}{0pt}

% Theorem and Proof setup
\newtheorem{theorem}{Theorem}
\newtheorem{lemma}{Lemma}
\newtheorem{definition}{Definition}

% Title Section
\title{MATD02 - Exploration of Finite Geometry }
\author{Danielle Voznyy}
\date{}

\begin{document}

    \maketitle

    \begin{abstract}
        Lorem ipsum dolor sit amet, consectetur adipiscing elit.
        Pellentesque pretium ut neque ac lacinia.
        Aenean ultricies dapibus ultricies.
        Nam tristique posuere quam, eget varius lorem posuere a.
        Nam hendrerit magna ex, vehicula tempor quam maximus at.
        Lorem ipsum dolor sit amet, consectetur adipiscing elit.
        Nullam vitae ultrices nulla, at dapibus lectus.
        Nullam ultrices eros eu lorem laoreet elementum.
        Nullam ullamcorper in lectus vel varius.
    \end{abstract}


    \section{Introduction}

    Finite geometries are axiomatically defined finite sets of point and lines.
    Set theory helps us provide a more rigorous definition of these elements,
    we'll use the following simple definitions going forward:

    \begin{itemize}
        \setlength\itemsep{0em}
        \item Points are elements in a finite set \( S \).
        \item Lines are elements in a subset of the power set of \( S \).
        \item We say a point \( p \) is on a line \( l \) when \( p \in l \).
        \item Lines \( l_1 \) and \( l_2 \) intersect when \( l_1 \cap l_2 \neq \emptyset \)..
    \end{itemize}

    Let's explore what axioms could lead to such a geometry.


    \section{Finite planes}

    \subsection{Finite affine planes}

    \subsection{Finite projective planes}

    Study fano plane

    \begin{theorem}
        Ex. Each point is on exactly three lines
    \end{theorem}

    \begin{proof}

    \end{proof}

    \subsection{Order of finite planes}


    \section{Finite geometries in higher dimensions}

%    Synthetic geometry studies such axiomatically defined geometries, and gives us tools to talk about
%    higher dimensional spaces with set theoretic definitions of dimension.

    Notice we didn't have to talk about our points as being two-dimensional,
    however something about our choice of points let us represent our spaces as two-dimensional planes.

    In fact, when looking at projective geometry through the lens of infinite sets like $R$,
    we get a definition of \enquote{projective spaces} by looking at one-dimensional vector subspaces of a vector space like $R^n$.\cite{weisstein_projective_nodate}
%    The projective part of projective geometry can be thought of as adding points at infinity to such a vector space,
    Here, our dimension comes from a concept of coordinates in the underlying vector space.


    In the finite case however\textemdash especially working purely from axioms\textemdash we don't have such an obvious idea of dimension.
    We can, however, generalize concepts from vector spaces to get the following definitions.\cite{noauthor_projective_2024,beutelspacher_projective_2000}

%    A subspace of the projective space is a subset X, such that any line containing two points of X is a subset of X (that is, completely contained in X). The full space and the empty space are always subspaces.
    \begin{definition}
        A subspace of the projective space is a subset X, such that any line containing two points of X is a subset of X (that is, completely contained in X). The full space and the empty space are always subspaces.
    \end{definition}

    \begin{definition}
        The geometric dimension of the space $P$ is said to be n if that is the largest number for which there is a strictly ascending chain of subspaces of this form:
        \[\emptyset = X_{-1}\subset X_{0}\subset \cdots X_{n}=P.\]
    \end{definition}

    \begin{theorem}
        Projective planes are projective spaces of dimension 2.
    \end{theorem}

    \begin{proof}
    \end{proof}

    \subsection{Finite projective spaces of dimension $\geq 3$ }


    \section{Future Developments}

    \subsection{Conjectures}

    From wiki: Is the order of a finite plane always a prime power?

    % References
    \bibliography{references}
    \bibliographystyle{IEEEtran}
\end{document}
