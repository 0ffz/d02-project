\documentclass[12pt]{article}

% Packages for formatting and math
\usepackage{amsmath, amsthm, amssymb}
\usepackage{cite}
\usepackage{geometry}
\usepackage{csquotes}
\geometry{a4paper, margin=2.5cm}

%\setlength{\parskip}{\baselineskip}%
%\setlength{\parindent}{0pt}
\usepackage{parskip}
\usepackage{amssymb}
\usepackage{amsfonts}
\usepackage[colorinlistoftodos,prependcaption,textsize=tiny]{todonotes}
\setlength{\parindent}{0pt}

% Theorem and Proof setup
\newtheorem{theorem}{Theorem}
\newtheorem{lemma}{Lemma}
\newtheorem{definition}{Definition}

% Title Section
\title{MATD02 - Exploration of Finite Geometry }
\author{Danielle Voznyy}
\date{}

\begin{document}

    \maketitle

    \begin{abstract}
        We explore finite geometries through a synthetic point of view.
        Starting with axioms, we explore finite planes, their orders, and prove some properties about the Fano plane.
        We then look at how concepts like dimension in projective spaces can be described without coordinate systems.
        Finally, we discuss some conjectures and future directions in the field.
    \end{abstract}


    \section{Finite plane geometry}

    Finite geometries provide a good introduction to axiomatic, sometimes called synthetic, definitions of geometries
    because they work with finite number of points that can help visualize more abstract concepts.

    We'll begin by describing finite planes using two sets and the set containment relation.

    \begin{definition}
        Let $P$ be nonempty a finite set and $L \subseteq \mathcal{P}(P)$, be a nonempty subset of the power set of $P$.

        We'll call an element of $P$ is a point in our space, and an element of $L$ is a line.
        We say a point $p$ is on a line when $p \in L$ and lines $L_1, L_2$ intersect when $L_1 \cap L_2 \neq \emptyset$
        \todo{Mention that there's more abstract definitions? We like to talk about planes and more in higher dimensions but unsure if relevant to this.}
        .
    \end{definition}

    We can now categorize these planes using axioms.
    Two broad types of geometries come from a similar set of axioms which differ in the existence of parallel lines.
    Affine geometry, where parallel lines exist, and projective geometry, where they do not.
    As it turns out, these are quite similar despite projective geometry being simpler to work with\cite[p.~7]{beutelspacher_projective_2000}.

    \subsection{Finite projective planes}

    Let's begin by defining projective spaces axiomatically, with an extra axiom for projective planes,

    \begin{definition}
        A projective space is a such that,
        \begin{itemize}
            \item For distinct points $p, q$, there is exactly one line $L$ such that ${p,q} \subseteq L$.
            We call it the line containing $p, q$, or just the line $pq$.
            \item Given points $a,b,c,d$, if $ab$ intersects $cd$, then $ac$ intersects $bd$
            \item Any line contains at least three elements.
            \item We call this projective space a projective plane if any two lines intersect.
        \end{itemize}
    \end{definition}

    \subsubsection{The fano plane}

    Study fano plane\todo[inline]{Not sure what the best way to introduce this is? Space for a quick proof about it containing 7 points, an image?}

    \subsection{Order of finite projective spaces}

    Finite projective spaces have an interesting property, all their lines contain an equal number of elements, which gives rise to an idea of order.

    \begin{theorem}
        Given a projective space, there exists a positive integer $q$ such that any line in $L$ contains exactly $q+1$ elements.
        We call q the order of this space.
    \end{theorem}

    \begin{proof}
        \todo[inline]{Prove.}
    \end{proof}

    \section{Dimension in finite projective spaces}

%    Synthetic geometry studies such axiomatically defined geometries, and gives us tools to talk about
%    higher dimensional spaces with set theoretic definitions of dimension.

    Notice to give our projective space an idea of being planes (somehow being two-dimensional), we had to add an axiom stating any two lines intersect.
    We might now wonder what this axiom actually does, and what the idea of dimension even means here.

    In fact, when looking at projective geometry through the lens of infinite sets like $\mathbb{R}$,
    we get a definition of projective spaces by looking at one-dimensional vector subspaces of a vector space like $\mathbb{R}^n$.\cite{weisstein_projective_nodate}
%    The projective part of projective geometry can be thought of as adding points at infinity to such a vector space,
    Here, our dimension comes from a concept of coordinates in the underlying vector space.

    In the finite case\textemdash especially working purely from axioms\textemdash we don't have such an obvious idea of dimension.
    We can, however, generalize concepts from vector spaces to get the following definitions.\cite{noauthor_projective_2024,beutelspacher_projective_2000}

    \begin{definition}
        A subspace of the projective space is a subset X, such that any line containing two points of X is a subset of X (that is, completely contained in X).
        The full space and the empty space are always subspaces.
    \end{definition}

    \begin{definition}
        The geometric dimension of the space $P$ is said to be n if that is the largest number for which there is a strictly ascending chain of subspaces of this form:
        \[\emptyset = X_{-1}\subset X_{0}\subset \cdots X_{n}=P.\]
    \end{definition}

    \begin{theorem}
        Projective planes are projective spaces of dimension 2.
    \end{theorem}

    \begin{proof}
        \todo[inline]{Prove.}
    \end{proof}

    \todo[inline]{I believe this definition can be described more intuitively as being able to draw continuous versions of our lines through points in n-dimensions without the lines intersecting.}

    \subsection{Finite projective spaces of dimension $\geq 3$ }

    As it turns out, order and dimension are powerful tools for classifying finite geometries.
    In fact, we use the notation $PG(n, q)$ to talk about projective spaces of dimension $n$ and order $q+1$
    that arise from a field of order $q$, written $GF(q)$ (since they are isomorphic up to order.)\todo{Quickly mention what isomorphism means for projective spaces.}

    The Veblen-Young theorem shows finite projective spaces of dimension $n \geq 3$ can always be described using such a field, up to isomorphism.
    However, finite projective planes can be constructed in more ways, specifically non-Desarguesian planes, which make
    it much harder to classify planes of order two\todo{Mention examples? State Desargues' theorem?}.

    \section{Future Developments}

    The difficulty in classifying finite projective planes gives rise to an open question:

    \textit{Is the order of a finite plane always a prime power?}

    \todo[inline]{Bruck–Ryser theorem}
    \todo[inline]{Classification up to order 10, why is it so difficulty to do past this small number?}

    % References
    \bibliography{references}
    \bibliographystyle{IEEEtran}
\end{document}
