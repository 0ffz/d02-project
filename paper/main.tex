\documentclass[12pt]{article}

% Packages for formatting and math
\usepackage{amsmath, amsthm, amssymb}
\usepackage{cite}
\usepackage{geometry}
\geometry{a4paper, margin=2.5cm}


% Theorem and Proof setup
\newtheorem{theorem}{Theorem}
\newtheorem{lemma}{Lemma}

% Title Section
\title{MATD02 - Exploration of Finite Geometry }
\author{Danielle Voznyy}
\date{}

\begin{document}

    \maketitle

    \begin{abstract}
        Lorem ipsum dolor sit amet, consectetur adipiscing elit.
        Pellentesque pretium ut neque ac lacinia.
        Aenean ultricies dapibus ultricies.
        Nam tristique posuere quam, eget varius lorem posuere a.
        Nam hendrerit magna ex, vehicula tempor quam maximus at.
        Lorem ipsum dolor sit amet, consectetur adipiscing elit.
        Nullam vitae ultrices nulla, at dapibus lectus.
        Nullam ultrices eros eu lorem laoreet elementum.
        Nullam ullamcorper in lectus vel varius.
    \end{abstract}

    \section{Introduction}

    Finite geometries are axiomatically defined finite sets of point and lines.
    Set theory helps us provide a more rigorous definition of these elements,
    we'll use the following simple definitions going forward:

    \begin{itemize}
        \setlength\itemsep{0em}
        \item Points are elements in a finite set \( S \).
        \item Lines are elements in a subset of the power set of \( S \).
        \item We say a point \( p \) is on a line \( l \) when \( p \in l \).
        \item Lines \( l_1 \) and \( l_2 \) intersect when \( l_1 \cap l_2 \neq \emptyset \)..
    \end{itemize}

    Let's explore what axioms could lead to such a geometry.

    \section{Finite planes}

    \subsection{Finite affine planes}

    \subsection{Finite projective planes}

    Study fano plane

    \begin{theorem}
        Ex. Each point is on exactly three lines
    \end{theorem}

    \begin{proof}

    \end{proof}


    \subsection{Order of finite planes}

    \section{Synthetic geometry and higher dimensions}

    Notice we didn't have to talk about our points as being two-dimensional,
    however something about the axioms we used let us represent our spaces as two dimensional planes.

    Synthetic geometry studies such axiomatically defined geometries, and gives us tools to talk about
    higher dimensional spaces with set theoretic definitions of dimension.

    DEFINE projective plane axiomatically

    DEFINE geometric dimension

    \subsection{Projective spaces and geometric dimension}

    Projective planes are two-dimensional projective spaces

    \begin{theorem}
    \end{theorem}

    \begin{proof}
    \end{proof}

    \subsection{Finite projective planes of order $\geq 3$ }

    \section{Future Developments}

    Take a look at notes from the group theory textbooks.

    \subsection{Conjectures}

    From wiki: Is the order of a finite plane always a prime power?

    % References
    \bibliography{references}
    \bibliographystyle{IEEEtran}
\end{document}
